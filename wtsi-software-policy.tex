\documentclass[10pt,a4paper]{article}
\usepackage[a4paper,hmargin=1.25in,vmargin=1in]{geometry}
\usepackage[utf8]{inputenc}
\usepackage[colorlinks]{hyperref}
\usepackage{verbatim}
\usepackage{listings}

%%%%%%%%%%%%%%%%%%%%%%%%%%%%%%%%%%%%%%%%
% WTSI Software Policy
%%%%%%%%%%%%%%%%%%%%%%%%%%%%%%%%%%%%%%%%
\begin{document}

\title{
Wellcome Trust Sanger Institute \\
Software Policy\\
DRAFT
}
\author{WTSI Informatics Committee}
\date{December 2014}

\maketitle

% title of responsible executive
\newcommand{\exectitle}[0]{Chief Operating Officer}

% name of responsible executive
\newcommand{\execperson}[0]{Martin Dougherty} 

% fixed-width (typewriter) face for filenames
\newcommand{\filename}[1]{\texttt{#1}} 

% minipage for boilerplate text
\newenvironment{boilerplate}[1][]
  {\minipage{\linewidth}% \begin{filecode}[#1]
   \lstset{basicstyle=\ttfamily\footnotesize,breaklines=false,frame=shadowbox,rulesepcolor=\color{blue},#1}}
   {\endminipage}% \end{filecode}

% paragraph indentation
\setlength{\parindent}{0pt} % Default is 15pt.

% vertical space between paragraphs (standard 4mm +/- 2mm)
\setlength{\parskip}{4mm plus2mm minus2mm}

% each section starts a new page
\let\stdsection\section
\renewcommand\section{\newpage\stdsection}




%%%%%%%%%%%%%%%%%%%%%%%%%%%%%%%%%%%%%%%%
%%%%%%%%%%%%%%%%%%%%%%%%%%%%%%%%%%%%%%%%
%%
%% Policy requirements
%%
%%%%%%%%%%%%%%%%%%%%%%%%%%%%%%%%%%%%%%%%
%%%%%%%%%%%%%%%%%%%%%%%%%%%%%%%%%%%%%%%%
\section{Policy requirements}
\label{section:policy}

%%%%%%%%%%%%%%%%%%%%%%%%%%%%%%%%%%%%%%%%
% Copyright assignment
%%%%%%%%%%%%%%%%%%%%%%%%%%%%%%%%%%%%%%%%
\subsection{Assign copyright to Genome Research Ltd}
\label{section:policy.copyright}
\par It is a condition of employment and of being granted visiting worker status 
at the Wellcome Trust Sanger Institute (the Institute) that software created in the 
course of work with the Institute is considered work made for hire and must have 
copyright in the source code assigned to Genome Research Ltd (GRL) unless it 
falls under a general or specific exception (see section \ref{section:exceptions}). 
A copyright notice must be included along with all source code indicating that 
GRL is a copyright holder (normally this would be included near the top of each 
source code file in a comment block; see section \ref{section:impnotes.copyright} 
for best pratices on how to indicate copyright). 

% Scope
\par This condition applies whether the employee or visiting worker (staff) 
created the software while physically located on the Institute's premises or 
elsewhere, whether using the Institute's equipment or other equipment, and 
whether the software is intended only for private use or is planned to be 
distributed to others (including being released publicly). GRL has a potential 
authorship interest in any software created in the course of work with the 
Institute, so unless staff are certain that software they develop falls 
under an exception, they are advised to either assign copyright in all software 
they write to GRL or to obtain a specific exception to retain or assign copyright 
otherwise.


%%%%%%%%%%%%%%%%%%%%%%%%%%%%%%%%%%%%%%%%
% Public release in advance of publication
%%%%%%%%%%%%%%%%%%%%%%%%%%%%%%%%%%%%%%%%
\subsection{Release publicly in advance of publication}

\par Software under GRL copyright that directly supports published work authored 
by GRL employees (e.g. software implementing methods described in a methods 
paper or software presented in an application note) must be released publicly 
using an acceptable free software license (see section \ref{section:policy.licenses}) 
by the time of the related publication at the very latest. 

\par Useful and potentially reusable software under GRL copyright that 
represents a substantial portion of the analysis and/or processing that went into 
producing results presented in a research paper should also be released 
publicly using an acceptable license by the time the research paper is 
published. 


%%%%%%%%%%%%%%%%%%%%%%%%%%%%%%%%%%%%%%%%
% Release using an acceptable free software license
%%%%%%%%%%%%%%%%%%%%%%%%%%%%%%%%%%%%%%%%
\subsection{Release using an acceptable free software license}
\label{section:policy.licenses}
\par Software under GRL copyright may be distributed outside of 
the Institute under any free software license as defined by the Free 
Software Foundation (FSF). A rough definition of free software is that it gives 
the users ``the freedom to run, copy, distribute, study, change and improve 
the software,'' but for a more complete definition please refer to the FSF 
(\url{http://www.gnu.org/philosophy/free-sw.html}). An incomplete list of free 
software licenses (which are acceptable) as well as nonfree software licenses 
(which are not acceptable) can be found at: 
\url{http://www.gnu.org/licenses/license-list.html}. Other licenses may be 
acceptable as well, provided they meet the requirements listed in the Free 
Software Foundation's definition. 

\par Prior to distributing a particular piece of software outside the Institute for the 
first time (including sending privately to an external collaborator or developing 
code in public such as by using \url{http://github.com/} or 
\url{http://sourceforge.net/} for source control), the original author(s) should, in 
conjunction with their line management, choose a free software license and apply 
it to the source code according to the best practices for that particular license 
(see section \ref{section:impnotes.recommended} for a set of recommended 
licenses along with best practices for applying them). Staff do not have to choose 
one of the recommended licenses, but if they do not, they must make sure that it 
is a free software license, that they understand the requirements of the license, 
and that it is correctly applied to the source code in a manner consistent with 
this policy. 

\par Staff must make each subsequent releases of a particular piece of software 
under the same license as the previous release, unless the terms of the previous 
release's license would permit the recipients to redistribute modified code under a 
different license, in which case a subsequent release may be distributed by staff 
under that license (i.e. it is acceptable to change to a license compatible with 
the previous license and staff are not more restricted in this regard than those 
they convey the software to). The reason for this restriction is primarily to prevent 
staff with limited or no involvement in a software project from releasing that 
software under terms that might conflict with the intentions of the original 
authors or undermine potential commercialisation opporuntities for GRL. 

\par Any staff member(s) who wish to re-release software which is under GRL 
copyright under a new license not compatible with the previous release should 
apply to the IC for permission to release it under a different license (this would 
include other acceptable free software licenses as well as alternative licenses 
that otherwise would not be acceptable, including nonfree licensing 
agreements). The IC's primary interest in considering such an application 
should be to ensure that the proposed license is appropriate and that it serves 
well the interests of the Institute and the wider community. The IC should also 
ensure that all relevant stakeholders (including staff involved in the creation of 
the software) are consulted and will involve the \exectitle\ if necessary to 
ascertain whether the change of license could effect commercialisation 
arrangements. 

\par A significant advantage for staff members in releasing code publicly and 
under free licenses is that it guarantees indefinite access to their work, even 
if they move to another institute. There have been notable examples in 
academia where individuals have been unable to continue working on software 
that they wrote because of licensing restrictions, and this policy avoids that 
outcome by allowing all software to be made publicly available under a free 
software license. 


%%%%%%%%%%%%%%%%%%%%%%%%%%%%%%%%%%%%%%%%
% Respect copyright and protect GRL copyright
%%%%%%%%%%%%%%%%%%%%%%%%%%%%%%%%%%%%%%%%
\subsection{Respect copyright and protect GRL copyright}
\label{section:policy.protection}

\par Staff must respect the copyright interests of others, in particular when producing 
software which falls under GRL copyright. Were GRL copyrighted code to be found 
to be infringing on the copyright of others, reputational damage to the Institute could 
occur, and the copyright status of other software under GRL copyright may be called 
into question. 

\par In general, it is good development practice to reuse code when possible, and in 
many cases there are good free software libraries that are available for 
incorporation into other software. The use of such libraries is encouraged -- however, 
staff should not take code from other sources and incorporate it into software under 
GRL copyright (whether by direct copying, some derivation or translation, or even 
linking), unless:
\begin{enumerate}
\item it is allowed under the terms of license offered by the copyright holders; and
\item all requirements of that license are met. 
\end{enumerate} 

\par Due to a lack of a principle of fair use that applies world-wide, staff should not 
rely upon any fair use provisions of copyright law as the basis on which to 
incorporate the works of others in whole or in part into GRL copyrighted software. 

\par In addition, staff should not do anything to undermine the copyright position of GRL 
with respect to code to which they contribute as works made for hire. For example, 
staff are not permitted to dedicate works made for hire for the Institute into the 
public domain as that would undermine GRLs copyright interest in that work. 
Likewise, staff must not incorporate any code for which the copyright status is 
not clear into GRL copyrighted software and should assume that any software 
that contains neither a copyright notice nor a public domain dedication is copyrighted 
material that must not be used. 

\par Staff should also keep records of all individual authors who have contributed to 
each source file in order to assist GRL with any legal challenges to the copyright 
status of the work. If the software is distributed, the authorship records should be 
distributed along with it (see section \ref{section:impnotes.records} for best practices 
in this regard). 



%%%%%%%%%%%%%%%%%%%%%%%%%%%%%%%%%%%%%%%%
%%%%%%%%%%%%%%%%%%%%%%%%%%%%%%%%%%%%%%%%
%%
%% Policy exceptions
%%
%%%%%%%%%%%%%%%%%%%%%%%%%%%%%%%%%%%%%%%%
%%%%%%%%%%%%%%%%%%%%%%%%%%%%%%%%%%%%%%%%
\section{Policy exceptions}
\label{section:exceptions}

%%%%%%%%%%%%%%%%%%%%%%%%%%%%%%%%%%%%%%%%
% General exceptions
%%%%%%%%%%%%%%%%%%%%%%%%%%%%%%%%%%%%%%%%
\subsection{General exceptions to copyright assignment}
\label{section:exceptions.general}
Section \ref{section:policy.copyright} requires that staff assign copyright in 
software they create to GRL. This section describes general exceptions to 
that requirement which do not require any special permission outside of this 
policy itself. 

\begin{itemize}

\item Software completely unrelated to work with the Institute

\par If staff develop software in their free time which is not in any way connected 
with their work at the Institute or with the business of the institute, it would 
probably not be considered a work made for hire and therefore would not be 
subject to this policy. Whether or not a particular software project would or 
would not be considered a work made for hire is a complex legal question and 
is outside the scope of this policy. Staff who are uncertain whether software 
they are developing or have developed is a work made for hire should seek 
legal advice or obtain a specific exception (see section \ref{section:exceptions.special}) 
for the avoidance of any doubt. 

\item Uncopyrightable code

\par This policy does not apply to code which does not rise to the level of a 
creative work or which would not be copyrightable for some other reason 
(such as having previously been dedicated into the public domain by the author). 

\par One example might be contributing extremely minor modifications to an existing 
codebase such as correcting minor spelling or syntax issues or adjusting whitespace. 
It could also include making the same extremely minor change repeatedly across a 
large number of lines and/or files, such as running a simple find-and-replace 
algorithm across an entire codebase to correct a typographical or spelling error. 
Generally, it is the degree of creativity involved and not the amount of code changed 
that determines whether a particular set of modifications is copyrightable. 


\item Small bug fixes and enhancements to third-party software

\par Staff may, in the course of their work, have occasion to modify software 
developed by third parties. It is often desirable that changes be contributed 
back to the third party maintainer so that others can benefit from the modifications. 
In many cases, this can be done within the contraints of this policy by simply 
assigning copyright in the modified code to GRL and providing the modified code 
under one of the acceptable licenses \ref{section:policy.licenses}. 

\par However, in some cases the third-party may not be willing to accept contributions 
without assigning copyright to them (this is likely to be the policy in the case of 
corporate-owned software but is also common practice for many open source 
foundations, including the FSF). To make it easy for staff to contribute small 
(but copyrightable) changes back to the community from which we receive 
software, a general exception to the requirement to assign copyright to GRL is 
hereby granted for small contributions made for the purpose of fixing bugs, 
improving performance, or contributing minor enhancements to a third-party 
software package for which the maintainer would otherwise not accept the 
modified code contribution. 

\par In general, ``small'' contributions should consist of a set of relatively minor 
contributions that togehter should not consitute more than 100 lines of code 
and should not have taken more than one working day of development effort 
to produce. Several distinct contributions can be made by a particular staff 
member to a single piece of third-party software, but the combined total across 
all such contributions to that software package should still be considered small. 
In other words, there is a lifetime limit for each member of staff under this general 
exception to contribute up to 100 lines of code (or one working day of 
development effort) to each individual third-party software package, beyond 
which a specific exception should be obtained. 

\par Staff are authorised to self-certify that they are allowed to contribute code 
under the terms of this clause (e.g. staff can write a declaration that says 
something to the effect of: ``I wrote this code and am authorised to contribute 
it to the project''). However, if a third-party requires a document signed 
by the employer, staff should obtain a specific exception (see 
section \ref{section:exceptions.special}) and ask for a letter signed by the \exectitle.

\end{itemize}


%%%%%%%%%%%%%%%%%%%%%%%%%%%%%%%%%%%%%%%%
% Specific exceptions
%%%%%%%%%%%%%%%%%%%%%%%%%%%%%%%%%%%%%%%%
\subsection{Specific exceptions to copyright assignment}
\label{section:exceptions.special}

\par If a staff member wishes to develop software in the general area of his/her work 
but does not want to assign copyright to GRL, this should be discussed in the first 
instance with an IC member -- ideally before beginning work on the project. The IC 
member can then bring the request for a specific exception to the IC for a decision. 

\par Any staff member wishing to obtain a specific exception to the requirement to 
assign copyright to GRL must receive written permission from the IC before 
distributing any software outside the requirements of this policy. The IC will consult 
with the \exectitle\ in these matters as appropriate. 

\par The written permission should indicate the specific work(s) and staff member(s) 
to which it applies as well as either a blanket copyright disclaimer or a specific list of 
the organisation(s) or individual(s) to whom it is acceptable to assign copyright in 
lieu of assigning it to GRL. Permission in the form of an email from the IC secretary 
would suffice as written permission for the purposes of this policy, although it is also 
possible to request a signed letter from the \exectitle\ if that is required (e.g. for a 
third-party organisation's records). 





%%%%%%%%%%%%%%%%%%%%%%%%%%%%%%%%%%%%%%%%
%%%%%%%%%%%%%%%%%%%%%%%%%%%%%%%%%%%%%%%%
%%
%% Implementation notes
%%
%%%%%%%%%%%%%%%%%%%%%%%%%%%%%%%%%%%%%%%%
%%%%%%%%%%%%%%%%%%%%%%%%%%%%%%%%%%%%%%%%
\section{Implementation notes}
\label{section:impnotes}

%%%%%%%%%%%%%%%%%%%%%%%%%%%%%%%%%%%%%%%%
% How to indicate copyright
%%%%%%%%%%%%%%%%%%%%%%%%%%%%%%%%%%%%%%%%
\subsection{How to indicate copyright}
\label{section:impnotes.copyright}
\par To assign copyright in software source code to GRL, a copyright notice should 
be included within its source code (normally within a comment block in the first 
few lines of each file). If a piece of software is comprised of multiple source code 
files, then each individual file must have a separate copyright notice. In order to 
not risk undermining the legal status of the copyright, it is a requirement that the 
copyright notice include the year(s) in which copyrightable contributions were 
made to the file, and it is important not to assert copyright for any year in which 
no copyrightable contributions were made to that file (because that could risk the 
copyright status of the whole file). The only exception to this is when starting a 
new file, it is acceptable to begin it with a copyright notice for the current year, in 
anticipation of the initial version of the file being created. This makes it easier to 
work with template headers that can be copied into place to start a new source 
file. If it later turns out that copyrightable contributions are not made until 
a later year, simply change the copyright notice to reflect the year(s) in which 
copyrightable contributions were actually made (i.e. replace the year from the 
template with the actual year when the file contents are added). 

\par The copyright notice should be of the form: 
\begin{verbatim}
Copyright (C) <YEARS> Genome Research Ltd.
\end{verbatim}

\par Where `\texttt{<YEARS>}' should include the list of years in which copyrightable 
contributions were made to the software. Each such year should be listed individually 
and separated by commas. A range of years (separated by dashes) should ideally 
not be used as it tends to lead to confusion and mistakes. If a range of years is used 
in any source file, a notice must be included in the documentation that indicates that 
the range of years should be interpreted as being equivalent to a list of years from the 
first to the last and including each consecutive year in between 
(see appendix \ref{appendix:range.of.years}). 

\subsubsection{Modifications to existing code}
\par Whenever existing source files are modified in a way which includes copyrightable 
contributions (e.g. adding more than a few lines of new code or substantially changing 
existing code), staff must ensure that a GRL copyright notice is included in the file 
and that the year in which those modifications were created is listed in that notice. 

\par If an existing source file already contains a GRL copyright notice, this can be 
accomplished by simply appending the year in which modifications were created 
to the end of the list of years in the existing copyright notice. If that year is already 
listed, it is not necessary to make any changes to the notice. 

\par If an existing source file does not contain any GRL copyright notice, add a 
new line following any existing copyright notices and including the year in which 
the modifications were created. If the existing source file contains no copyright 
notice, unless it is absolutely clear what the copyright status is of the work, staff 
should assume it is actually copyrighted by someone and therefore it should not 
be used as to do so would risk infringing that unknown copyright when the 
combined work is distributed. One scenario in which no copyright notice would be 
present is if the work has been dedicated to the public domain by the author. In 
that instance, please ensure that a public domain dedication is present in the file. 
If the file is clearly in the public domain but it does not say so in the file (i.e. because 
staff have communicated directly with the author, or the public domain dedication was 
listed on the download site), add a comment to the file indicating that portions of it 
are based on that earlier work which was placed into the public domain (give the 
name of the author and include a copy of their written public domain dedication). 

\par Note that it is important to comply with the terms of license under 
which original source files are received from external sources (see 
\ref{section:policy.protection}). In some cases that may limit 
under which license(s) it is acceptable to distribute a combined work  
including that original source code and other code. In some instances, it may not  
be possible to combine arbitary third-party code into a combined work without 
violating the licensing terms of at least one of them (for example, GPLv2 code 
cannot be combined with Apache 2.0 code), so it is important to be aware 
of any license restrictions before starting to use third-party source code or libraries. 
Note that it is possible to combine material available under multiple different 
third-party licenses provided they are all compatible with the license under which 
the combined work is distributed.

%\url{http://www.softwarefreedom.org/resources/2007/gpl-non-gpl-collaboration.html}


%%%%%%%%%%%%%%%%%%%%%%%%%%%%%%%%%%%%%%%%
% Individual authorship records
%%%%%%%%%%%%%%%%%%%%%%%%%%%%%%%%%%%%%%%%
\subsection{Individual authorship records}
\label{section:impnotes.records}
\par Each staff member who makes copyrightable contributions to a source file 
should include their own name in the file, usually on an ``Author'' line in the 
same comment block as the copyright notice and license text). Staff members 
may also choose to include an email address to aid in identifying and contacting 
them in the future, should any issues arise regarding the copyright status of the 
work. In addition, it is good practice to use a source control system (such as git 
or svn) that allows individual line-level contributions to also be attributed to 
specific individuals. If contributions are accepted from third-parties, they 
should also indicate their authorship within each file (in addition to any 
copyright notices they may add as appropriate).

\par The purpose of these records is twofold: 
\begin{itemize}
\item to assist with any legal challenges to the copyright status of the code within 
each file (see section \ref{section:policy.protection})
\item to recognise the set of individuals who contributed to the work, in much the 
same way an author list on a peer-reviewed paper acknowledges the efforts of the 
researchers who worked on the paper
\end{itemize}

\par For both purposes, it is important that these records offer a faithful 
representation of the set of individuals who contributed to the work in each 
individual file and should never be modified except to correct actual errors. 

\par Licenses chosen for distribution should require  recipients to preserve
authorship records in any versions they redistribute. 


%%%%%%%%%%%%%%%%%%%%%%%%%%%%%%%%%%%%%%%%
% Application of license
%%%%%%%%%%%%%%%%%%%%%%%%%%%%%%%%%%%%%%%%
\subsection{Applying a license}

\par In addition to the copyright notices which are required to be included in each 
source code file regardless of whether or not software is ever distributed outside 
the institute, any software that is distributed outside the Institute must have 
one of the acceptable free software licenses applied to any code under GRL 
copyright so that it is clear to recipients what the license terms are under which 
they are receiving the software. 

\par A statement must be added to each source code file following the copyright 
notice and author record and indicating under which license(s) the code is 
available and under what terms it can be copied, modified, and redistributed. 
In the case of short licenses (including any of the recommended permissive 
licenses) the best way to do that is to include the full text of the license in each 
source file. Other licenses (including all of the recommended copyleft licenses) 
have a much longer license document which would not be reasonable to include 
within each source file. For such licenses, a copying permission statement should 
be included in each source file in place of the actual license, and that statement 
should indicate what license applies and where the licensee can find a copy of 
that license. The authors of the license will normally provide an appropriate 
statement or template to be used for that purpose.

\par In addition, it is good practice to take additional steps to make the license 
terms clear to users who may not be accustomed to reading the source code 
to software they obtain: 
\begin{itemize}
\item state on any web pages providing access to the software under which 
license it is available

\item provide the full text of the license in the root directory of any distribution 
(including a source code or binary distribution) -- usually this is included in a file 
called \filename{LICENSE} or \filename{COPYING}

\end{itemize}

\subsubsection{Boilerplate}

\par Boilerplate templates that can be used (with the addition of appropriate 
comment characters) to begin a new source code file under each of the 
recommended licenses (see section \ref{section:impnotes.recommended}) can 
be found in Appendix \ref{appendix:boilerplate}. For other free software licenses, 
use the general format of these boilerplate (i.e. copyright notice first, then 
author records, then license instructions) but follow the guidelines for applying 
the license as specified by the license authors. 


%%%%%%%%%%%%%%%%%%%%%%%%%%%%%%%%%%%%%%%%
% Recommended licenses
%%%%%%%%%%%%%%%%%%%%%%%%%%%%%%%%%%%%%%%%
\subsection{Recommended licenses}
\label{section:impnotes.recommended}

\par Different free software licenses impose different restrictions on how recipients 
of the software can use, modify, and redistribute the code. 

\par All of the free software licenses we recommend share the requirement that 
a recipient of the code (the licensee) must maintain copyright, attribution, and disclaimer 
statements in any versions of the source code and binaries which they redistribute. 
They also all include a warranty/liability disclaimer, basically saying that the software 
is provided without any warranty and that it isn't the fault of the copyright holders if 
the software causes any unintention harm to the licensee. Licenses that share these 
properties should be chosen whenever possible. 

\subsubsection{Permissive licenses}
\label{section:impnotes.recommended.permissive}
\par Some licenses (permissive licenses) don't include many additional restrictions on 
the licensee. In practice, this means that under a permissive license the licensee is 
allowed to take the code, combine it with their own software, modify it to include 
additional functionality, and sell it to others under another license terms (including 
nonfree licenses) without sharing the source code of the software (or their changes) 
with anyone. 

\par These licenses may be a good choice when:
\begin{itemize}
\item there is little or no potential for paid-license commercialisation (because being available under a permissive license undermines 
the business case for a paid license), or when the benefits of being permissive outweigh 
the benefits from any potential commercialisation; and 
\item substantial development is not expected to occur outside of the Institute, and if it does, it is acceptable that modified versions may not be made available. 
\end{itemize}
Because the license terms are 
more permissive, software under permissive licenses might tend to get more use 
(especially within companies) than software under more restrictive free licenses, 
so it might be a good choice if the most important factor is to have the software used 
as widely as possible (for example, software which serves as the reference 
implementation for a data format standard). 

\par Permissive licenses that are recommended for use include: 
\begin{itemize}
\item Apache License, Version 2.0 (Apache) (\url{https://www.gnu.org/licenses/license-list.html#apache2})
\item Expat License (MIT) (\url{https://www.gnu.org/licenses/license-list.html#Expat})
\item Modified BSD license (mBSD) (\url{https://www.gnu.org/licenses/license-list.html#ModifiedBSD})
\end{itemize}

The MIT license and mBSD license are quite similar permissive licenses except that the 
mBSD license explicitly prohibits the use of the name of the copyright holder to endorse 
or promote products derived from the software without permission. However, that explicit 
statement is probably not required as permission would be required to claim any 
endorsement, so functionally the MIT and mBSD licenses are equivalent. Both of those 
licenses are so short that the full license text should be included near the top of each and 
every source code file (see Appendix \ref{appendix:boilerplate} and \ref{appendix:examples} 
for boilerplate templates and examples). These licenses are compatible with most other 
free software licenses (including the recommended copyleft licenses; see section 
\ref{section:impnotes.recommended.copyleft}) -- this means that a licensee of software 
conveyed to them under MIT or mBSD licenses can be redistributed under the terms of 
another license such as the GPL. 

The Apache license is also fairly permissive, but offers additional protections to the 
in the form of patent termination and indemnification clauses. This basically means 
that the copyright holders automatically grant the licensee a royalty-free patent license 
such that using the software would never be considered patent infringement. It also 
terminates those automatic patent licenses if the licensee files a lawsuit against anyone 
claiming that the software infringes on their patents. The Apache license is therefore 
a good choice for a permissive license, although it is important to note that those 
additional provisions make it less compatible with other licenses. In particular, it is 
not possible to combine code under a GPLv2 license (without the "or later version" 
clause) and the Apache license, whereas this can be done with the MIT or mBSD 
licenses. 

\subsubsection{Copyleft licenses}
\label{section:impnotes.recommended.copyleft}
\par Other licenses (copyleft licenses) place additional restrictions on what the 
licensee is allowed to do, generally in the interest of ensuring that the code, and 
any modifications to the code, continues to stay free. 

\par Three common copyleft restrictions are: 

\begin{enumerate}
\item the licensee is additionally required to distribute any modified versions of the 
source code under the same license terms (weak copyleft) \label{copyleft.weak}
\item the licensee is required to adopt the same license for any additional code that it 
is combined with to create a single piece of software (strong copyleft) \label{copyleft.strong}
\item the licensee is additionally required to distribute the source code which 
corresponds to the version of the software running on a network server to all 
users interacting with the software remotely through a computer network (if the 
modified software supports such interaction) at no charge (strong network copyleft) \label{copyleft.strong.network}
\end{enumerate}

If the \ref{copyleft.weak}st requirement is required but the \ref{copyleft.strong}nd and 
\ref{copyleft.strong.network}rd requirements are \textbf{not}, a licensee would be free 
to use the code as a component of a larger project (e.g. by linking) and could distribute 
that larger project as a whole under any terms they wish. However, if modifications are 
made to the actual code conveyed under these terms, the licensee would have to make 
those modifications available to anyone to whom they convey the source code. 
Choosing this combination of options is most often done when the project is a software 
library (e.g. biojava, bioperl) and the authors wish to ensure all library code is distributed 
under the same license, but want the library as a whole to be as widely used as possible, 
so do not want to restrict what software it is used as a component of. 

To enforce only the \ref{copyleft.weak}st requirement, we recommend:
\begin{itemize}
\item GNU Lesser General Public License (LGPL) version 3 (or later) (\url{https://www.gnu.org/licenses/license-list.html#LGPL})
\end{itemize}

To enforce the \ref{copyleft.weak}st and \ref{copyleft.strong}nd requirements but not the 
\ref{copyleft.strong.network}rd, (such as is done by many high profile projects such as 
mysql, the Linux kernel, GCC, the GNU science library, and many self contained Sanger 
projects), we recommend:
\begin{itemize}
\item GNU General Public License (GPL) version 3 (or later) (\url{https://www.gnu.org/licenses/license-list.html#GNUGPL})
\end{itemize}

To provide additional protections against licensees modifying the software and not sharing 
those modifications with end users, and especially if the software involves user interaction 
over a network (such as an application with a web-based user-interface), the 
\ref{copyleft.strong.network}rd requirement may be appropriate to require as well. 
In that case we recommend:
\begin{itemize}
\item GNU Affero General Public License (AGPL) version 3 (or later) (\url{https://www.gnu.org/licenses/license-list.html#AGPL})
\end{itemize}

\par It is safe to use the AGPL on software that does not actually involve any remote 
user interaction if it is desirable to ensure that any future use of the software that does provide 
remote user interaction will be required to provide any modifications that have been made 
to the source code to the users of that service. For example, software for genomic analysis 
which runs only on a single machine might be installed into a cloud-based genomic analysis 
service and offered to end users for data analysis. If licensed under the AGPL, the 
cloud service would be required to provide the full source code for the version of software 
that they provide for their customers to run (including any modifications they have made) 
to any customers who interacts with the service to analyse data. If the genomic analysis 
software had been licensed under GPL or a less restrictive license, the cloud service would 
be permitted to modify the software and to run their modified version on customer data 
without providing them with the source code for those changes or even informing them of 
what changes have been made. 




%%%%%%%%%%%%%%%%%%%%%%%%%%%%%%%%%%%%%%%%
% Translation & commercialisation
%%%%%%%%%%%%%%%%%%%%%%%%%%%%%%%%%%%%%%%%
\subsection{Translation \& commercialisation considerations}

\par None of the free software licenses are exclusive licenses, so it 
is possible for copyright holder(s) to license software under 
a nonfree commercial license that is also available publicly under a free 
software license, although in that case the more restrictive copyleft 
licenses (such as the GPL or AGPL) may offer a better business
proposition for a ``paid license'' form of commercialisation than would 
a more permissive license (such as mBSD or Apache). On the other hand, 
the more permissive licenses may tend to result in more widespread use, 
especially within commercial organisations, and that could result in a 
larger potential customer base for a ``paid support'' style of commercialisation. 

\par If software includes code with copyright from multiple parties 
(e.g. if GRL copyright applies only to some portions of the software but other 
organisations and/or individuals also hold a copyright in some portions), the 
license can only be changed to an incompatible license by agreement of all 
copyright holders. Staff interested in potential commercialisation of GRL 
software after an initial public release should think carefully about what 
license to choose in order to best facilitate potential commercialisation 
opportunities, as well as the potential need for project-level policies 
regarding accepting contributions from third-parties (e.g. they may want to 
ask any third-party contributors to assign copyright in their contributions to 
GRL). 


%%%%%%%%%%%%%%%%%%%%%%%%%%%%%%%%%%%%%%%%
% Old code
%%%%%%%%%%%%%%%%%%%%%%%%%%%%%%%%%%%%%%%%
\subsection{Code not conforming to these conditions}
\label{section:impnotes.oldcode}

\subsubsection{Software written entirely by staff}
\label{section:impnotes.oldcode.grlcopyright}
\par Any software written entirely by staff that is distributed publicly (e.g. on the 
Institute's web or ftp sites, on Institute-affliated GitHub repositories, etc) 
should contain GRL copyright notices and an acceptable free software license 
as required by this policy (unless a specific exception has been granted). 
Copyright notices that are incorrect (e.g. staff claimed a personal copyright 
when in fact the copyright is owned by GRL) should be corrected. Please 
note that it is possible that some staff have historically had employment 
contracts that do not require them to assign copyright, and in that case 
personal copyright notices must be preserved (see section 
\ref{section:impnotes.oldcode.thirdparty}).

\par If the latest publicly available version of existing software is one in which  
copyright notices and/or licensing statements have been omitted from some or 
all source code files, then this policy should be applied to that software and 
the notices and license text added as appropriate to bring it into line with this 
policy (it is acceptable to make a new release of the software that is compliant 
while keeper older noncompliant releases available for download for 
historical/archival purposes). 

\par Importantly, the year(s) included in a copyright notice should reflect the 
actual year(s) in which copyrightable code was created -- it is not acceptable 
to simply add the current year to a file unless a new creative contribution has 
been made to it. Use any available information (e.g. source control records, 
release notes, etc) to ascertain the original year(s) during which substantive 
contributions were made to each file as well as the authors involved (contact 
them if possible). It is generally better to err on the side of including an earlier 
copyright year rather than a later year if there is any doubt as to the actual year(s) 
involved (i.e. if the original date the file was created is known, but it is not clear 
when it was modified, include at least that original year of creation). 

\par If the latest release does not include any license whatsoever, it is 
acceptable for staff to treat it as if it had not yet been publicly released, decide 
on an initial license, and release a new version under that license 
(see section \ref{section:policy.licenses}). 

\par If the software includes a license which is not acceptable or contains 
multiple incompatible licenses, IC should be informed and should decide an 
appropriate license to apply to the software as a whole. 

\subsubsection{Software including third-party code}
\label{section:impnotes.oldcode.thirdparty}

\par Software which is not compliant with the requirements in this policy but 
which includes code with a copyright held by third-parties needs to be handled 
carefully. If it can be determined what contributions were written by staff as 
opposed to third parties, any missing GRL copyright notices should be added 
alongside third-party copyright notices. 

\par The license status of each piece of code not under GRL copyright should 
be determined, and the entire package should only be released under a license 
that is compatible with the terms of all individual licenses. If the licenses are 
incompatible, further distribution of the software should be stopped until the 
situation can be resolved. It may be necessary to remove some component of 
the software in order to legally release such software, although in some cases 
it may also be possible to contact the copyright holder for one or more of the 
components and obtain a different license for the code. 


%%%%%%%%%%%%%%%%%%%%%%%%%%%%%%%%%%%%%%%%
% Risks of noncompliance
%%%%%%%%%%%%%%%%%%%%%%%%%%%%%%%%%%%%%%%%

\subsection{Risks of non-compliance}
\label{section:policy.copyright.risks}
\par Failing to comply with this policy may put the Institute's legal position with 
respect to its intellectual property at risk and could undermine our ability to protect 
the substantial investment we have made in software development, including 
the protection of works made available under free software licenses and those 
involved in commercial partnerships. 

\par In addition, distribution of software by staff in a manner not consistent with 
this policy could be considered an infringement of GRL's copyright interest in the 
work (for which the Institute would be within its rights to seek damages). Worse, 
distribution of software including a copyright notice claiming a personal copyright 
or assigning copyright to another organisation when in fact the software is a work 
made for hire for the Institute would be a misrepresentation of the facts and could 
potentially be considered a form of fraud. In that case anyone harmed by the false  
statement (such as other contributors to the software or those who have received 
and used it under an invalid license) could potentially sue the responsible staff 
member for damages. Staff should also be aware that making false copyright 
claims is illegal in some jurisdictions, and so knowingly doing so could expose 
the staff member to the possibility of criminal prosecution. 




\appendix

%%%%%%%%%%%%%%%%%%%%%%%%%%%%%%%%%%%%%%%%
% Appendix A: Boilerplate for recommended licenses
%%%%%%%%%%%%%%%%%%%%%%%%%%%%%%%%%%%%%%%%
\section{Boilerplate}
\label{appendix:boilerplate}

The following are boilerplate headers that should be used, substituting 
DATES, PROGRAM-NAME, PROGRAM-AUTHOR and EMAIL as appropriate.

Note: Including the email address (EMAIL) is optional. 

\subsection{GPL license (for software consisting of multiple source files)}

\begin{boilerplate}
\lstinputlisting{boilerplate/gpl-multi.txt}
\end{boilerplate}

\subsection{GPL license (for software consisting of only one source file)}
\begin{boilerplate}
\lstinputlisting{boilerplate/gpl-single.txt}
\end{boilerplate}

\subsection{LGPL license (for software consisting of multiple source files)}
\begin{boilerplate}
\lstinputlisting{boilerplate/lgpl-multi.txt}
\end{boilerplate}

\subsection{LGPL license (for software consisting of only one source file)}
\begin{boilerplate}
\lstinputlisting{boilerplate/lgpl-single.txt}
\end{boilerplate}

\subsection{AGPL license (for software consisting of multiple source files)}
\begin{boilerplate}
\lstinputlisting{boilerplate/agpl-multi.txt}
\end{boilerplate}

\subsection{AGPL license (for software consisting of only one source file)}
\begin{boilerplate}
\lstinputlisting{boilerplate/agpl-single.txt}
\end{boilerplate}

\subsection{Modified-BSD license}
\begin{boilerplate}
\lstinputlisting{boilerplate/mbsd.txt}
\end{boilerplate}



%%%%%%%%%%%%%%%%%%%%%%%%%%%%%%%%%%%%%%%%
% Appendix B: documentation boilerplate
%%%%%%%%%%%%%%%%%%%%%%%%%%%%%%%%%%%%%%%%
\section{Note regarding range of years to include in documention as needed}
\label{appendix:range.of.years}
The following boilerplate should be included somewhere in the 
documentation distributed with the software if any files contain copyright 
notices that use a ``range of years'' syntax in place of a comma-delimited 
list of years. This note need not be at the start of the a documentation file, 
but must be included somewhere in the documentation. 

\begin{boilerplate}
\lstinputlisting{docnotes/range-of-years.txt}
\end{boilerplate}


%%%%%%%%%%%%%%%%%%%%%%%%%%%%%%%%%%%%%%%%
% Appendix C: Examples
%%%%%%%%%%%%%%%%%%%%%%%%%%%%%%%%%%%%%%%%

\section{Examples}
\label{appendix:examples}

\subsection{GPL example with a single author}
\begin{boilerplate}
\lstinputlisting{examples/gpl-single-author.txt}
\end{boilerplate}




%%%%%%%%%%%%%%%%%%%%%%%%%%%%%%%%%%%%%%%%
% Document history
%%%%%%%%%%%%%%%%%%%%%%%%%%%%%%%%%%%%%%%%
\section*{Document history}

\begin{tabular}{|c|c|p{6cm}|}
\hline
\textbf{version (date)}	& \textbf{author}	& \textbf{changes} \\
\hline
v1 (11 Oct 2006)		& Richard Durbin	& \parbox{6cm}{
} \\
\hline
v2 (20 Sep 2007)		& Tim Hubbard		& \parbox{6cm}{
} \\
\hline
v3 (3 Dec 2009)		& Tim Hubbard		& \parbox{6cm}{
} \\
\hline
v4 (23 Dec 2010) 		& Tim Hubbard		& \parbox{6cm}{
clarifications: policy statement; section 1 (LGPL license description). \\
amendments: Appendix 3. (Modified-BSD license text) \\
} \\
\hline
v5 (4 May 2012)		& Tim Hubbard		& \parbox{6cm}{
clarifications: policy statement. \\
amendments: AGPL licenses added, Appendix 3. (boilerplate instructions) \\
} \\
\hline
v6 (18 July 2014)		& Joshua Randall	& \parbox{6cm}{
amendments: updated Director of Corporate Services. \\
formatting: converted to LaTeX, consolidated version/author history. \\
} \\
\hline
v7 (proposed draft)		& Joshua Randall	& \parbox{6cm}{
major substantive changes: simplification of policy to be less prescriptive\\
restructured into core policy requirements along with implementation notes. 
} \\
\hline
\end{tabular}


\end{document}
