\documentclass[10pt,a4paper]{article}
\usepackage[a4paper,hmargin=1.25in,vmargin=1in]{geometry}
\usepackage[utf8]{inputenc}
\usepackage[colorlinks]{hyperref}
\usepackage{verbatim}
\usepackage{listings}

%%%%%%%%%%%%%%%%%%%%%%%%%%%%%%%%%%%%%%%%
% WTSI Software Policy
%%%%%%%%%%%%%%%%%%%%%%%%%%%%%%%%%%%%%%%%
\begin{document}

\title{
Wellcome Trust Sanger Institute \\
Software Policy
}
\author{WTSI Informatics Committee}
\date{December 2014}

\maketitle

% title of responsible executive
\newcommand{\exectitle}[0]{Chief Operating Officer}

% name of responsible executive
\newcommand{\execperson}[0]{Martin Dougherty} 

% fixed-width (typewriter) face for filenames
\newcommand{\filename}[1]{\texttt{#1}} 

% minipage for boilerplate text
\newenvironment{boilerplate}[1][]
  {\minipage{\linewidth}% \begin{filecode}[#1]
   \lstset{basicstyle=\ttfamily\footnotesize,breaklines=false,frame=shadowbox,rulesepcolor=\color{blue},#1}}
   {\endminipage}% \end{filecode}

% paragraph indentation
\setlength{\parindent}{0pt} % Default is 15pt.

% vertical space between paragraphs (standard 4mm +/- 2mm)
\setlength{\parskip}{4mm plus2mm minus2mm}

% each section starts a new page
\let\stdsection\section
\renewcommand\section{\newpage\stdsection}




%%%%%%%%%%%%%%%%%%%%%%%%%%%%%%%%%%%%%%%%
%%%%%%%%%%%%%%%%%%%%%%%%%%%%%%%%%%%%%%%%
%%
%% Policy requirements
%%
%%%%%%%%%%%%%%%%%%%%%%%%%%%%%%%%%%%%%%%%
%%%%%%%%%%%%%%%%%%%%%%%%%%%%%%%%%%%%%%%%
\section{Policy requirements}
\label{section:policy}

%%%%%%%%%%%%%%%%%%%%%%%%%%%%%%%%%%%%%%%%
% Copyright assignment
%%%%%%%%%%%%%%%%%%%%%%%%%%%%%%%%%%%%%%%%
\subsection{Assign copyright to Genome Research Ltd}
\label{section:policy.copyright}
\par It is a condition of employment and of being granted visiting worker status at the 
Wellcome Trust Sanger Institute (the Institute) that software created in the course 
of work with the Institute is considered work made for hire and must have 
copyright in the source code assigned to Genome Research Ltd (GRL) unless it 
falls under a general or specific exception (see section \ref{section:exceptions}). 
A copyright notice must be included along with all source code indicating that 
GRL is a copyright holder (normally this would be included near the top of each 
source code file in a comment block; see section \ref{section:howto.copyright} for best 
pratices on how to indicate copyright). 

% Scope
\par This condition applies whether the employee or visiting worker (staff) 
created the software while physically located on the Institute's premises or 
elsewhere, whether using the Institute's equipment or other equipment, and 
whether the software is intended only for private use or is planned to be 
distributed to others (including being released publicly). GRL has a potential 
authorship interest in any software created in the course of work with the 
Institute, so unless staff are certain that software they develop falls 
under an exception, they are advised to either assign copyright in all software 
they write to GRL or to obtain a specific exception to retain or assign copyright 
otherwise.


%%%%%%%%%%%%%%%%%%%%%%%%%%%%%%%%%%%%%%%%
% Public release in advance of publication
%%%%%%%%%%%%%%%%%%%%%%%%%%%%%%%%%%%%%%%%
\subsection{Release publicly in advance of publication}

\par Software that supports published work authored by GRL employees 
(e.g. software implementing methods described in a methods paper, software 
presented in an application note, software used in analysis and/or data 
processing for a research paper describing the results of those analyses and 
processing, etc.) must be released publicly using an acceptable license 
(see section \ref{section:policy.licenses}) by the time of the related publication 
at the very latest. 


%%%%%%%%%%%%%%%%%%%%%%%%%%%%%%%%%%%%%%%%
% Acceptable licenses
%%%%%%%%%%%%%%%%%%%%%%%%%%%%%%%%%%%%%%%%
\subsection{Release using an acceptable free software license}
\label{section:policy.licenses}
\par Software under GRL copyright may be distributed outside of 
the Institute under any free software license as defined by the Free 
Software Foundation (FSF). A rough definition of free software is that it gives 
the users ``the freedom to run, copy, distribute, study, change and improve 
the software,'' but for a more complete definition please refer to the FSF 
(\url{http://www.gnu.org/philosophy/free-sw.html}). An incomplete list of free 
software licenses (which are acceptable) as well as nonfree software licenses 
(which are not acceptable) can be found at: 
\url{http://www.gnu.org/licenses/license-list.html}. Other licenses may be 
acceptable as well, provided they meet the requirements listed in the Free 
Software Foundation's definition. 

\par Prior to distributing a particular piece of software outside the Institute for the 
first time (including sending privately to an external collaborator or uploading code 
to public source control repositories such as \url{http://github.com/} 
or \url{http://sourceforge.net}), the original author(s) should, in conjunction with their 
line management, choose a free software license and apply it to the source code 
according to the best practices for that particular license 
(see section \ref{section:impnotes.recommended} for a set of recommended 
licenses along with best practices for applying them). Staff do not have to choose 
one of the recommended licenses, but if they do not, they must make sure that it 
is a free software license, that they understand the requirements of the license, 
and that it is correctly applied to the source code in a manner consistent with 
this policy. 

\par Staff must make each subsequent releases of a particular piece of software 
under the same license as the previous release, unless the terms of the previous 
release's license would permit the recipients to redistribute modified code under a 
different license, in which case a subsequent release may be distributed by staff 
under that license (i.e. it is acceptable to change to a license compatible with 
the previous license and staff are not more restricted in this regard than those 
they convey the software to). The reason for this restriction is primarily to prevent 
staff with limited or no involvement in a software project from releasing that 
software under terms that might conflict with the intentions of the original 
authors or undermine potential commercialisation opporuntities for GRL. 

\par Any staff member(s) who wish to re-release software which is under GRL 
copyright under a new license not compatible with the previous release should 
apply to the IC for permission to release it under a different license (this would 
include other acceptable free software licenses as well as alternative licenses 
that otherwise would not be acceptable, including nonfree licensing 
agreements). The IC's primary interest in considering such an application 
should be to ensure that the proposed license is appropriate and that it serves 
well the interests of the Institute and the wider community. The IC should also 
ensure that all relevant stakeholders (including staff involved in the creation of 
the software) are consulted and will involve the \exectitle\ if necessary to 
ascertain whether the change of license could effect commercialisation 
arrangements. 

\par A significant advantage for staff members in releasing code publicly and 
under free licenses is that it guarantees indefinite access to their work, even 
if they move to another institute. There have been notable examples in 
academia where individuals have been unable to continue working on software 
that they wrote because of licensing restrictions, and this policy avoids that 
outcome by allowing all software to be made publicly available under a free 
software license. 


%%%%%%%%%%%%%%%%%%%%%%%%%%%%%%%%%%%%%%%%
% Respect copyright and protect GRL copyright
%%%%%%%%%%%%%%%%%%%%%%%%%%%%%%%%%%%%%%%%
\subsection{Respect copyright and protect GRL copyright}
\label{section:policy.protection}

\par Staff must respect the copyright interests of others, in particular when producing 
software which falls under GRL copyright. Were GRL copyrighted code to be found 
to be infringing on the copyright of others, reputational damage to the Institute could 
occur, and the copyright status of our other software may be called into question. 

\par In general, it is good development practice to reuse code when possible, and in 
many cases there are good free software libraries that are available for 
incorporation into other software. The use of such libraries is encouraged -- however, 
staff should not take code from other sources and incorporate it into software under 
GRL copyright (whether by direct copying, some derivation or translation, or even 
linking), unless:
\begin{enumerate}
\item it is allowed under the terms of license offered by the copyright holders; and
\item all requirements of that license are met. 
\end{enumerate} 

\par Due to a lack of a principle of fair use that applies world-wide, staff should not 
rely upon any fair use provisions of copyright law as the basis on which to 
incorporate the works of others in whole or in part into GRL copyrighted software. 

\par In addition, staff should not do anything to undermine the copyright position of GRL 
with respect to code to which they contribute as works made for hire. For example, 
staff are not permitted to dedicate works made for hire for the Institute into the 
public domain as that would undermine GRLs copyright interest in that work. 
Likewise, staff must not incorporate any code for which the copyright status is 
not clear into GRL copyrighted software and should assume that any software 
that contains neither a copyright notice nor a public domain dedication is copyrighted 
material that must not be used. 


\par Staff should also keep records of the individual authors who have contributed to 
each source file in order to assist GRL with any legal challenges to the copyright 
status of the work. If the software is distributed, the authorship records should be 
distributed along with it (see section \ref{section:impnotes.records} for best practices 
in this regard). 



%%%%%%%%%%%%%%%%%%%%%%%%%%%%%%%%%%%%%%%%
%%%%%%%%%%%%%%%%%%%%%%%%%%%%%%%%%%%%%%%%
%%
%% Policy exceptions
%%
%%%%%%%%%%%%%%%%%%%%%%%%%%%%%%%%%%%%%%%%
%%%%%%%%%%%%%%%%%%%%%%%%%%%%%%%%%%%%%%%%
\section{Policy exceptions}
\label{section:exceptions}

%%%%%%%%%%%%%%%%%%%%%%%%%%%%%%%%%%%%%%%%
% General exceptions
%%%%%%%%%%%%%%%%%%%%%%%%%%%%%%%%%%%%%%%%
\subsection{General exceptions to copyright assignment}
\label{section:exceptions.general}
Section \ref{section:policy.copyright} requires that staff assign copyright in 
software they create to GRL. This section describes general exceptions to 
that requirement which do not require any special permission outside of this 
policy itself. 

\begin{itemize}

\item Software completely unrelated to work with the Institute

\par If staff develop software in their free time which is not in any way connected 
with their work at the Institute or with the business of the institute, it would 
probably not be considered a work made for hire and therefore would not be 
subject to this policy. Whether or not a particular software project would or 
would not be considered a work made for hire is a complex legal question and 
is outside the scope of this policy. Staff who are uncertain whether software 
they are developing or have developed is a work made for hire should seek 
legal advice or obtain a specific exception (see section \ref{section:exceptions.special}) 
for the avoidance of any doubt. 

\item Uncopyrightable code

\par This policy does not apply to code which does not rise to the level of a 
creative work or which would not be copyrightable for some other reason 
(such as having previously been dedicated into the public domain by the author). 

\par One example might be contributing extremely minor modifications to an existing 
codebase such as correcting minor spelling or syntax issues or adjusting whitespace. 
It could also include making the same extremely minor change repeatedly across a 
large number of lines and/or files, such as running a simple find-and-replace 
algorithm across an entire codebase to correct a typographical or spelling error. 
Generally, it is the degree of creativity involved and not the amount of code changed 
that determines whether a particular set of modifications is copyrightable. 


\item Small bug fixes and enhancements to third-party software

\par Staff may, in the course of their work, have occasion to modify software 
developed by third parties. It is often desirable that those changes be contributed 
back to the third party maintainer so that others can benefit from the modifications. 
In many cases, this can be done within the contraints of this policy by simply 
assigning copyright in the modified code to GRL and providing the modified code 
under one of the acceptable licenses \ref{section:policy.licenses}. 

\par However, in some cases the third-party may not be willing to accept contributions 
without assigning copyright to them (this is likely to be the policy in the case of 
corporate-owned software but is also common practice for many open source 
foundations, including the FSF). To make it easy for staff to contribute small 
(but copyrightable) changes back to the community from which we receive 
software, a general exception to the requirement to assign copyright to GRL is 
hereby granted for small contributions made for the purpose of fixing bugs, 
improving performance, or contributing minor enhancements to a third-party 
software package for which the maintainer would otherwise not accept the 
modified code contribution. 

\par In general, ``small'' contributions should consist of a set of relatively minor 
contributions that togehter should not consitute more than 100 lines of code 
and should not have taken more than one working day of development effort 
to produce. Several distinct contributions can be made by a particular staff 
member to a single piece of third-party software, but the combined total across 
all such contributions to that software package should still be considered small. 
In other words, there is a lifetime limit for each member of staff under this general 
exception to contribute up to 100 lines of code (or one working day of 
development effort) to each individual third-party software package, beyond 
which a specific exception should be obtained. 

\end{itemize}


%%%%%%%%%%%%%%%%%%%%%%%%%%%%%%%%%%%%%%%%
% Specific exceptions
%%%%%%%%%%%%%%%%%%%%%%%%%%%%%%%%%%%%%%%%
\subsection{Specific exceptions to copyright assignment}
\label{section:exceptions.special}

\par If a staff member wishes to develop software in the general area of his/her work 
but does not want to assign copyright to GRL, this should be discussed in the first 
instance with an IC member -- ideally before beginning work on the project. The IC 
member can then bring the request for a specific exception to the IC for a decision. 

\par Any staff member wishing to obtain a specific exception to the requirement to 
assign copyright to GRL must receive written permission from the IC before 
distributing any software outside the requirements of this policy. The IC will consult 
with the \exectitle\ in these matters as appropriate. 

\par The written permission should indicate the specific work(s) and staff member(s) 
to which it applies as well as either a blanket copyright disclaimer or a specific list of 
the organisation(s) or individual(s) to whom it is acceptable to assign copyright in 
lieu of assigning it to GRL. Permission in the form of an email from the IC secretary 
would suffice as written permission for the purposes of this policy, although it is also 
possible to request a signed letter from the \exectitle\ if that is required (e.g. for a 
third-party organisation's records). 






%%%%%%%%%%%%%%%%%%%%%%%%%%%%%%%%%%%%%%%%
%%%%%%%%%%%%%%%%%%%%%%%%%%%%%%%%%%%%%%%%
%%
%% Implementation notes
%%
%%%%%%%%%%%%%%%%%%%%%%%%%%%%%%%%%%%%%%%%
%%%%%%%%%%%%%%%%%%%%%%%%%%%%%%%%%%%%%%%%
\section{Implementation notes}
\label{section:impnotes}

%%%%%%%%%%%%%%%%%%%%%%%%%%%%%%%%%%%%%%%%
% How to indicate copyright
%%%%%%%%%%%%%%%%%%%%%%%%%%%%%%%%%%%%%%%%
\subsection{How to indicate copyright}
\label{section:howto.copyright}
\par To assign copyright in software source code to GRL, a copyright notice should 
be included within its source code (normally within a comment block in the first 
few lines of each file). If a piece of software is comprised of multiple source code 
files, then each individual file must have a separate copyright notice. In order to 
not risk undermining the legal status of the copyright, it is a requirement that the 
copyright notice include the year(s) in which copyrightable contributions were 
made to the file, and it is important not to assert copyright for any year in which 
no copyrightable contributions were made to that file (because that could risk the 
copyright status of the whole file). The only exception to this is when starting a 
new file, it is acceptable to begin it with a copyright notice for the current year, in 
anticipation of the initial version of the file being created. This makes it easier to 
work with template headers that can be copied into place to start a new source 
file. If it later turns out that copyrightable contributions are not made until 
a later year, simply change the copyright notice to reflect the year(s) in which 
copyrightable contributions were actually made (i.e. replace the year from the 
template with the actual year when the file contents are added). 

\par The copyright notice should be of the form: 
\begin{verbatim}
Copyright (C) <YEARS> Genome Research Ltd.
\end{verbatim}

\par Where `\texttt{<YEARS>}' should include the list of years in which copyrightable 
contributions \cite{define:copyrightable} were made to the software. Each such year 
should be listed individually and separated by commas. A range of years (separated 
by dashes) should ideally not be used as it tends to lead to confusion and mistakes. 
If a range of years is used in any source file, a notice must be included in the 
documentation that indicates that the range of years should be interpreted as being 
equivalent to a list of years from the first to the last and including each consecutive 
year in between (see appendix \ref{appendix:range.of.years}). 

\subsubsection{Modifications to existing code}
\par Whenever existing source files are modified in a way which includes copyrightable 
contributions (e.g. adding more than a few lines of new code or substantially changing 
existing code), staff must ensure that a GRL copyright notice is included in the file 
and that the year in which those modifications were created is listed in that notice. 

\par If an existing source file already contains a GRL copyright notice, this can be 
accomplished by simply appending the year in which your modifications were created 
to the end of the list of years in the existing copyright notice. If that year is already 
listed, it is not necessary to make any changes to the notice (but you should update 
authorship records). 

\par If an existing source file does not contain any GRL copyright notice, add a 
new line following any existing copyright notices and including the year in which 
the modifications were created. If the existing source file contains no copyright 
notice, unless it is absolutely clear what the copyright status is of the work, staff 
should assume it is actually copyrighted by someone and therefore it should not 
be used as to do so would risk infringing that unknown copyright when the 
combined work is distributed. One scenario in which no copyright notice would be 
present is if the work has been dedicated to the public domain by the author. In 
that instance, please ensure that a public domain dedication is present in the file. 
If the file is clearly in the public domain but it does not say so in the file (i.e. because 
staff have communicated directly with the author, or the public domain dedication was 
listed on the download site), add a comment to the file indicating that portions of it 
are based on that earlier work which was placed into the public domain (give the 
name of the author and include a copy of their written public domain dedication). 

\par Note that it is important to comply with the terms of license under 
which original source files were received, and in some cases that may limit 
under which license(s) it is acceptable to distribute a combined work  
including that original source code and other code. In some instances, it may not  
be possible to combine arbitary third-party code into a combined work without 
violating the licensing terms of at least one of them, so it is important to be aware 
of any license restrictions before starting to use third-party source code or libraries. 
Note that it is possible to combine material available under multiple different 
third-party licenses provided they are all compatible with the license under which 
the combined work is distributed.



%%%%%%%%%%%%%%%%%%%%%%%%%%%%%%%%%%%%%%%%
% Individual authorship records
%%%%%%%%%%%%%%%%%%%%%%%%%%%%%%%%%%%%%%%%
\subsubsection{Individual authorship record}
\label{section:impnotes.records}
\par Each staff member who makes copyrightable contributions to a source file 
should include their own name in the file, usually on an ``Author'' line in the 
same comment block as the copyright notice and license text). In addition, it is 
good practice to use a source control system (such as git or svn) that allows 
individual line-level contributions to also be attributed to specific individuals. If 
contributions are accepted from third-parties, they should also indicate their 
authorship within each file (in addition to any copyright notices they may be 
required to add).

\par The purpose of these records is twofold: 
\begin{itemize}
\item to assist with any legal challenges to the copyright status of the code within 
each file (see section \ref{section:policy.protection})
\item to recognise the set of individuals who contributed to the work, in much the 
same way an author list on a peer-reviewed paper acknowledges the efforts of the 
researchers who worked on the paper
\end{itemize}

\par For both purposes, it is important that these records offer a faithful 
representation of the set of individuals who contributed to the work in each 
individual file and should never be modified except to correct actual errors. 

\par Licenses chosen for distribution should require  recipients to preserve
authorship records in any versions they redistribute. 


%%%%%%%%%%%%%%%%%%%%%%%%%%%%%%%%%%%%%%%%
% Recommended licenses
%%%%%%%%%%%%%%%%%%%%%%%%%%%%%%%%%%%%%%%%
\subsection{Recommended licenses}
\label{section:impnotes.recommended}

\par Different free software licenses impose different restrictions on how recipients 
of the software can use, modify, and redistribute the code. 

\par Most (but not all) free software licenses require the user of the code (the licensee) 
to maintain copyright, attribution, and disclaimer statements in any versions of the 
source code and binaries which they redistribute. These licenses should be chosen 
whenever possible in order that authorship records (attribution) be preserved. 

\subsubsection{Permissive licenses}
\par Some licenses (permissive licenses) don't include any additional restrictions on 
the licensee. That means that a piece of software licensed under a permissive 
license allowed the licensee can take the code, combine it with their own software, 
modify it to include additional functionality, and sell it to others under a nonfree 
license without sharing the source code of the software (or their changes) with anyone. 
These licenses are a good choice when there is little or no potential for paid-license 
commercialisation (because being available under a permissive license undermines 
the business case for a paid license), or when the benefits of being permissive outweigh 
the benefits from any potential commercialisation. Because the license terms are 
more permissive, software under permissive licenses might tend to get more use 
(especially without companies) than software under more restrictive free licenses, 
so it might be a good choice if the most important factor is to have the software used 
as widely as possible (for example, software which serves as the reference 
implementation for a data format standard). 

\par Permissive licenses that are recommended for use include: 
\begin{itemize}
\item 3-clause modified BSD (mBSD) license 
\item X11 license
\item MIT license
\item Apache 2.0 license
\end{itemize}

\subsubsection{Copyleft licenses}
\par Other licenses (copyleft licenses) place additional restrictions on what the 
licensee is allowed to do, generally in the interest of ensuring that the code and 
any modifications to the code, continues to stay free. 

The choice between alternative licenses depends on 
what additional obligations are to be placed on the licensee. 

There are three additional obligations to consider: 

\begin{enumerate}
\item The licensee is additionally required to distribute any modified versions of the 
source code under the same license terms. 
\item The licensee is required to adopt the same license for any additional code that it 
is combined with to create a single piece of software. 
\item The licensee is additionally required to distribute the source code which 
corresponds to the version of the software running on a network server to all 
users interacting with the software remotely through a computer network (if the 
modified software supports such interaction) at no charge. 
\end{enumerate}

If the 1st and 3rd obligations are \textbf{not} required, a licensee is free to take the code, modify 
it and redistribute it under any terms that they wish, including commercially, only 
maintaining attribution and disclaimers to the original authors. This may not be a 
problem, since it does not interfere with any continued distribution of code by GRL, 
however it may mean external code improvements cannot be incorporated by GRL into 
its original code. Reasons for not requiring this obligation could be when you wish the 
code to be as widely adopted as possible, including within companies and when you do 
not fear a commercial competitor. The \textbf{Modified-BSD license} (as used by X11 
consortium) is appropriate in this case. Ensembl uses a variant of this license, which 
includes additional clauses related to the Ensembl name which is a registered trade 
mark. 
 
If the 1st obligation is required but the 2nd and 3rd obligations are \textbf{not}, a licensee is free 
to use the code as a component of a larger project but is free to distribute that larger 
project as a whole under what ever terms they wish. Choosing this combination of 
options is useful in cases where the project is a software library (e.g. biojava, bioperl) 
and the authors wish to ensure all library code is distributed under the same license, 
but want the library as a whole to be as widely used as possible, so do not want to 
restrict what software it is used as a component of. To enforce only the 1st obligation, 
the \textbf{LGPL license} is appropriate. This license obliges a licensee to enable the use of the 
licensee's program with whatever version of the LGPLed library a user sees fit, 
including one that that user has just modified themself. 

To enforce the 1st and 2nd obligation but not the 3rd, (such as is done by many high 
profile projects such as mysql and many self contained Sanger projects), the \textbf{GPL 
license} is appropriate. If your software provides user interaction over a network (such 
as an application with a web-based user-interface) then you may also consider requiring 
the 3rd obligation, in which case the \textbf{AGPL license} may be appropriate. Unless your 
project fits with one of the situations described above, the GPL (or AGPL) license is the 
most generally applicable license to use. It is also safe to use the AGPL on software that 
does not yet have any remote user interaction (e.g. a web interface) if you want to 
ensure that any future use of the software that does provide remote user interaction will 
be required to provide any modifications that have been made to your source code to 
the users of that service (this would not be required under the GPL if the modified 
version of your software is only run on a server but not distributed to others). 
 
The standard boilerplate header for each of these licenses is given in Appendix A, 
which in the case of LGPL, GPL and AGPL includes a link to the primary web site 
documenting the full license. Any of these licenses is acceptable as satisfying 
requirements above, so long as the copyright is assigned to GRL 
and the author(s) names are given.


%%%%%%%%%%%%%%%%%%%%%%%%%%%%%%%%%%%%%%%%
% Failure to comply
%%%%%%%%%%%%%%%%%%%%%%%%%%%%%%%%%%%%%%%%

\subsection{Failure to comply}
\label{section:policy.copyright.failure}
\par Failing to comply with this policy may put the Institute's legal position with respect 
to its intellectual property at risk and may undermine our ability to protect the 
substantial investment we have made in software development, including the protection 
of works made available under free software licenses. 

\par In addition, distribution of software in a manner not consistent with this policy could be 
considered an infringement of GRL's copyright interest in the work (for which the 
Institute could seek damages). Worse, distribution of software including a copyright 
notice claiming a personal copyright or assigning copyright to another 
organisation when in fact the software is a work made for hire for the Institute may 
be considered a form of fraud (in which case anyone harmed by the false 
statement could potentially sue for damages, and for which one may even be subject 
to criminal prosecution in some jurisdictions). 


%%%%%%%%%%%%%%%%%%%%%%%%%%%%%%%%%%%%%%%%
% Application of license
%%%%%%%%%%%%%%%%%%%%%%%%%%%%%%%%%%%%%%%%
\subsection{Applying a license}

Once you have chosen the license you wish to use, you must apply the license so that 
people can see it. For software that is made available outside the Institute, at a 
minimum you must:

\begin{itemize}
\item state on webpages providing access to the software which license is applied 
\item provide the full text of the license in the root directory of any source code 
distribution (usually in a file called \filename{LICENSE.TXT}) 
\item insert the appropriate header (created from one of standard boilerplates given in 
the Appendix A) at or near the start of each separate source code file. Since these 
boilerplates are legal statements it is important that the text is not edited in any 
way beyond substituting the required names and dates.
\item provide the full text of the license in the root directory of any non-source project 
distributions (usually in a file called \filename{LICENSE.TXT}). 
\item provide basic documentation in the root directory of any source code 
distribution, and include all clauses found in Appendix B in that documentation 
(usually in a file called \filename{README.TXT}). 
\end{itemize}


%%%%%%%%%%%%%%%%%%%%%%%%%%%%%%%%%%%%%%%%
% Translation & commercialisation
%%%%%%%%%%%%%%%%%%%%%%%%%%%%%%%%%%%%%%%%
\subsection{Translation \& commercialisation considerations}

\par None of the free software licenses are exclusive licenses, so it 
is possible for copyright holder(s) to license software under 
a nonfree commercial license that is also available publicly under a free 
software license, although in that case the more restrictive ``copyleft'' 
licenses (such as the GPL or AGPL) may offer a better business
proposition for a ``paid license'' form of commercialisation than would 
a more permissive license (such as mBSD or Apache). On the other hand, 
the more permissive licenses may tend to result in more widespread use, 
especially within commercial organisations, and that could result in a 
larger potential customer base for a ``paid support'' style of commercialisation. 

\par If software includes code with copyright from multiple parties 
(e.g. if GRL copyright applies only to some portions of the software but other 
organisations and/or individuals also hold a copyright in some portions), the 
license can only be changed to an incompatible license by agreement of all 
copyright holders. Staff interested in potential commercialisation of GRL 
software after an initial public release should think carefully about what 
license to choose in order to best facilitate potential commercialisation 
opportunities, as well as the potential need for project-level policies 
regarding accepting contributions from third-parties (e.g. they may want to 
ask any third-party contributors to assign copyright in their contributions to 
GRL). 


%%%%%%%%%%%%%%%%%%%%%%%%%%%%%%%%%%%%%%%%
% Old code
%%%%%%%%%%%%%%%%%%%%%%%%%%%%%%%%%%%%%%%%
\subsection{Code not conforming to these conditions}

\par Any code written entirely by staff that is distributed publicly (e.g. on the 
Institute's web or ftp sites, on Institute-affliated GitHub repositories, etc) 
should contain copyright notices and an acceptable free software license. 

\par If existing software is being distributed from which copyright notices 
and/or licensing statements have been omitted from some or all source 
code files, then this policy should be applied and the notices and license 
text added as appropriate. Importantly, the year(s) included in a copyright 
notice should reflect the actual year(s) in which copyrightable code was 
created -- it is not acceptable to simply add the current year to a file unless 
you make a new creative contribution to it. Use any available information 
(e.g. source control records, release notes, etc) to ascertain the original 
year(s) during which substantive contributions were made to each file as
well as the authors involved (contact them if possible). It is generally better 
to err on the side of including an earlier copyright year rather than a later 
year if there is any doubt as to the actual year(s) involved (i.e. if you know 
the original date the file was created but are not sure when it was modified, 
include at least that original year of creation). 

If the existing release does not include any license whatsoever, the IC should 
be informed and should decide an appropriate license to apply to the 
software. 


%%%%%%%%%%%%%%%%%%%%%%%%%%%%%%%%%%%%%%%%
% External projects
%%%%%%%%%%%%%%%%%%%%%%%%%%%%%%%%%%%%%%%%
\subsection{OLD*******Collaborations with others, and contributions to open source projects}

The following considerations apply when GRL employees or Sanger visiting/hosted 
scientists make substantial contributions to code maintained in collaboration with 
others outside the institute: 

\begin{itemize}
\item If the code is made publicly available under one of the standard acceptable open 
source licenses, or another license and header text that has been approved as 
described above, then again the line manager can approve contribution to the 
project, so long as a copyright statement is in place and the author's contribution 
is acknowledged. The copyright statement should include GRL as one of the 
copyright holders if a substantial fraction (say more than 25\% of the file) was 
written by a GRL employee or Sanger visiting/hosted scientist. 
\item If the code is developed as part of a research collaboration, and the software itself 
is a primary aim of the collaboration, then we would expect to require the 
software to be made widely available under an agreed license. If the 
collaboration is primarily about other matters, such as the generation and 
analysis of data, then it is acceptable for the software not to be distributed 
outside the collaboration, but even in this case a copyright statement and 
authorship attribution are required.
\item There may be cases where other arrangements are required, for example in 
collaborations with instrument manufacturers. In these cases a more formal 
agreement is required, which needs to be approved by the Director of Corporate 
Services who may consult the Wellcome Trust.
\item In some cases we use external software and either find and fix a bug, or make a 
useful modification, and want to feed that back to the author/copyright holder. 
Small bug fixes or enhancements of less than 100 lines or so of code are exempt 
from the provisions of this document. Larger contributions can also be made if 
authorised by an IC member. 
\end{itemize}


%%%%%%%%%%%%%%%%%%%%%%%%%%%%%%%%%%%%%%%%
% Signing-off 
%%%%%%%%%%%%%%%%%%%%%%%%%%%%%%%%%%%%%%%%
\subsection{OLD*******Contributions to open source projects that require ``signing off''}

Some open source projects require code to be ``signed-off'' with a declaration that says ``I 
wrote this code and am authorized to contribute it to the project'' (see section 5 in 
\url{http://linux.yyz.us/patch-format.html} for an example). The \exectitle\ can authorise 
code to be contributed in this fashion. 






\appendix

%%%%%%%%%%%%%%%%%%%%%%%%%%%%%%%%%%%%%%%%
% Appendix A: Boilerplate for recommended licenses
%%%%%%%%%%%%%%%%%%%%%%%%%%%%%%%%%%%%%%%%
\section{Boilerplate}
\label{appendix:boilerplate}

The following are boilerplate headers that should be used, substituting 
DATES, PROGRAM-NAME, PROGRAM-AUTHOR and EMAIL as appropriate.

Note: Including the email address (EMAIL) is optional. 

\subsection{GPL license (for software consisting of multiple source files)}

\begin{boilerplate}
\lstinputlisting{boilerplate/gpl-multi.txt}
\end{boilerplate}

\subsection{GPL license (for software consisting of only one source file)}
\begin{boilerplate}
\lstinputlisting{boilerplate/gpl-single.txt}
\end{boilerplate}

\subsection{LGPL license (for software consisting of multiple source files)}
\begin{boilerplate}
\lstinputlisting{boilerplate/lgpl-multi.txt}
\end{boilerplate}

\subsection{LGPL license (for software consisting of only one source file)}
\begin{boilerplate}
\lstinputlisting{boilerplate/lgpl-single.txt}
\end{boilerplate}

\subsection{AGPL license (for software consisting of multiple source files)}
\begin{boilerplate}
\lstinputlisting{boilerplate/agpl-multi.txt}
\end{boilerplate}

\subsection{AGPL license (for software consisting of only one source file)}
\begin{boilerplate}
\lstinputlisting{boilerplate/agpl-single.txt}
\end{boilerplate}

\subsection{Modified-BSD license}
\begin{boilerplate}
\lstinputlisting{boilerplate/mbsd.txt}
\end{boilerplate}



%%%%%%%%%%%%%%%%%%%%%%%%%%%%%%%%%%%%%%%%
% Appendix B: documentation boilerplate
%%%%%%%%%%%%%%%%%%%%%%%%%%%%%%%%%%%%%%%%
\section{Note regarding range of years to include in documention as needed}
\label{appendix:range.of.years}
The following boilerplate should be included somewhere in the 
documentation distributed with the software if any files contain copyright 
notices that use a ``range of years'' syntax in place of a comma-delimited 
list of years. This note need not be at the start of the a documentation file, 
but must be included somewhere in the documentation. 

\begin{boilerplate}
\lstinputlisting{docnotes/range-of-years.txt}
\end{boilerplate}


%%%%%%%%%%%%%%%%%%%%%%%%%%%%%%%%%%%%%%%%
% Appendix C: Examples
%%%%%%%%%%%%%%%%%%%%%%%%%%%%%%%%%%%%%%%%

\section{Examples}

\subsection{GPL example with a single author}
\begin{boilerplate}
\lstinputlisting{examples/gpl-single-author.txt}
\end{boilerplate}




%%%%%%%%%%%%%%%%%%%%%%%%%%%%%%%%%%%%%%%%
% Document history
%%%%%%%%%%%%%%%%%%%%%%%%%%%%%%%%%%%%%%%%
\section*{Document history}

\begin{tabular}{|c|c|p{6cm}|}
\hline
\textbf{version (date)}	& \textbf{author}	& \textbf{changes} \\
\hline
v1 (11 Oct 2006)		& Richard Durbin	& \parbox{6cm}{
} \\
\hline
v2 (20 Sep 2007)		& Tim Hubbard		& \parbox{6cm}{
} \\
\hline
v3 (3 Dec 2009)		& Tim Hubbard		& \parbox{6cm}{
} \\
\hline
v4 (23 Dec 2010) 		& Tim Hubbard		& \parbox{6cm}{
clarifications: policy statement; section 1 (LGPL license description). \\
amendments: Appendix 3. (Modified-BSD license text) \\
} \\
\hline
v5 (4 May 2012)		& Tim Hubbard		& \parbox{6cm}{
clarifications: policy statement. \\
amendments: AGPL licenses added, Appendix 3. (boilerplate instructions) \\
} \\
\hline
v6 (18 July 2014)		& Joshua Randall	& \parbox{6cm}{
amendments: updated Director of Corporate Services. \\
formatting: converted to LaTeX, consolidated version/author history. \\
} \\
\hline
v7 (proposed draft)		& Joshua Randall	& \parbox{6cm}{
major substantive changes: simplification of policy to be less prescriptive\\
restructured into core policy requirements along with implementation notes. 
} \\
\hline
\end{tabular}


\end{document}
