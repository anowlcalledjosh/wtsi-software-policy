\documentclass[10pt,a4paper]{article}
\usepackage[utf8]{inputenc}
\usepackage[colorlinks]{hyperref}
\usepackage{verbatim}


%%%%%%%%%%%%%%%%%%%%%%%%%%%%%%%%%%%%%%%%
% WTSI Software Policy
%%%%%%%%%%%%%%%%%%%%%%%%%%%%%%%%%%%%%%%%
\begin{document}

\title{
Wellcome Trust Sanger Institute \\
Software Policy
}
\author{WTSI Informatics Committee}
\date{July 2014}

\maketitle

% name of Director of Corporate Services
\newcommand{\docsperson}[0]{Martin Dougherty} 

% fixed-width (typewriter) face for filenames
\newcommand{\filename}[1]{\texttt{#1}} 

% smaller text size for boilerplate and keep on same page
\newenvironment{boilerplate}{\bgroup\small\samepage }{\egroup} 


%%%%%%%%%%%%%%%%%%%%%%%%%%%%%%%%%%%%%%%%
% Preamble
%%%%%%%%%%%%%%%%%%%%%%%%%%%%%%%%%%%%%%%%
\section*{WTSI policy on software copyright and licensing}
It is a condition of employment or visiting/hosted scientist status at the Wellcome Trust 
Sanger Institute (WTSI) that all software written as part of your work at the institute 
should have copyright assigned to Genome Research Ltd (GRL). 
 
Our standard policy is that we are happy to make GRL software publicly available with 
source code under a free software license (as defined by the free software foundation 
\url{http://www.fsf.org/licensing/licenses}), although other alternatives may be possible in 
particular circumstances. Generally, software supporting published analysis should be 
released by publication at the very latest. This document describes how this policy 
should be interpreted and implemented. 
 
A significant advantage for WTSI staff in releasing code in this way is that it guarantees 
them indefinite access to their work, even if they move institute. There have been 
notable examples in academia where individuals have been unable to continue working 
on software that they wrote because of licensing restrictions.


%%%%%%%%%%%%%%%%%%%%%%%%%%%%%%%%%%%%%%%%
% Choice of license
%%%%%%%%%%%%%%%%%%%%%%%%%%%%%%%%%%%%%%%%
\section{Choosing between acceptable licenses}

All acceptable licenses considered here require the user of the code (referred to 
subsequently as the licensee) to maintain attribution and disclaimers in any versions of 
the code which they redistribute. The choice between alternative licenses depends on 
what additional obligations you wish to place on the licensee. There are three additional obligations to consider: 

\begin{enumerate}
\item The licensee is additionally required to distribute any modified versions of the 
source code under the same license terms. 
\item The licensee is required to adopt the same license for any additional code that it 
is combined with to create a single piece of software. 
\item The licensee is additionally required to distribute the source code which 
corresponds to the version of the software running on a network server to all 
users interacting with the software remotely through a computer network (if the 
modified software supports such interaction) at no charge. 
\end{enumerate}

If the 1st and 3rd obligations are \textbf{not} required, a licensee is free to take the code, modify 
it and redistribute it under any terms that they wish, including commercially, only 
maintaining attribution and disclaimers to the original authors. This may not be a 
problem, since it does not interfere with any continued distribution of code by GRL, 
however it may mean external code improvements cannot be incorporated by GRL into 
its original code. Reasons for not requiring this obligation could be when you wish the 
code to be as widely adopted as possible, including within companies and when you do 
not fear a commercial competitor. The \textbf{Modified-BSD license} (as used by X11 
consortium) is appropriate in this case. Ensembl uses a variant of this license, which 
includes additional clauses related to the Ensembl name which is a registered trade 
mark. 
 
If the 1st obligation is required but the 2nd and 3rd obligations are \textbf{not}, a licensee is free 
to use the code as a component of a larger project but is free to distribute that larger 
project as a whole under what ever terms they wish. Choosing this combination of 
options is useful in cases where the project is a software library (e.g. biojava, bioperl) 
and the authors wish to ensure all library code is distributed under the same license, 
but want the library as a whole to be as widely used as possible, so do not want to 
restrict what software it is used as a component of. To enforce only the 1st obligation, 
the \textbf{LGPL license} is appropriate. This license obliges a licensee to enable the use of the 
licensee's program with whatever version of the LGPLed library a user sees fit, 
including one that that user has just modified themself. 

To enforce the 1st and 2nd obligation but not the 3rd, (such as is done by many high 
profile projects such as mysql and many self contained Sanger projects), the \textbf{GPL 
license} is appropriate. If your software provides user interaction over a network (such 
as an application with a web-based user-interface) then you may also consider requiring 
the 3rd obligation, in which case the \textbf{AGPL license} may be appropriate. Unless your 
project fits with one of the situations described above, the GPL (or AGPL) license is the 
most generally applicable license to use. It is also safe to use the AGPL on software that 
does not yet have any remote user interaction (e.g. a web interface) if you want to 
ensure that any future use of the software that does provide remote user interaction will 
be required to provide any modifications that have been made to your source code to 
the users of that service (this would not be required under the GPL if the modified 
version of your software is only run on a server but not distributed to others). 
 
The standard boilerplate header for each of these licenses is given in Appendix A, 
which in the case of LGPL, GPL and AGPL includes a link to the primary web site 
documenting the full license. Any of these licenses is acceptable as satisfying 
requirements above, so long as the copyright is assigned to Genome Research Limited 
and the author(s) names are given.

Given the legal nature of these boilerplate headers they must not be edited. If there is an 
additional license requirement that requires modification or addition to the text or the 
adoption of a different free software compatible license, this must be agreed with the 
Informatics Committee (IC) first and in consultation with the Director of Corporate 
Services (\docsperson).


%%%%%%%%%%%%%%%%%%%%%%%%%%%%%%%%%%%%%%%%
% Application of license
%%%%%%%%%%%%%%%%%%%%%%%%%%%%%%%%%%%%%%%%
\section{Applying the license}

Once you have chosen the license you wish to use, you must apply the license so that 
people can see it. For software that is made available outside the Institute, at a 
minimum you must:

\begin{itemize}
\item state on webpages providing access to the software which license is applied 
\item provide the full text of the license in the root directory of any source code 
distribution (usually in a file called \filename{LICENSE.TXT}) 
\item insert the appropriate header (created from one of standard boilerplates given in 
the Appendix A) at or near the start of each separate source code file. Since these 
boilerplates are legal statements it is important that the text is not edited in any 
way beyond substituting the required names and dates.
\item provide the full text of the license in the root directory of any non-source project 
distributions (usually in a file called \filename{LICENSE.TXT}). 
\item provide basic documentation in the root directory of any source code 
distribution, and include all clauses found in Appendix B in that documentation 
(usually in a file called \filename{README.TXT}). 
\end{itemize}


%%%%%%%%%%%%%%%%%%%%%%%%%%%%%%%%%%%%%%%%
% Old code
%%%%%%%%%%%%%%%%%%%%%%%%%%%%%%%%%%%%%%%%
\section{Old code not conforming to these conditions}

Any code written by GRL employees or Sanger visiting/hosted scientists that is 
distributed on our web or ftp site should contain an approved header statement as 
outlined above. If there is no such copyright and licensing statement now, or what 
there is does not comply, then this policy should be applied and the code amended as 
appropriate. 


%%%%%%%%%%%%%%%%%%%%%%%%%%%%%%%%%%%%%%%%
% External projects
%%%%%%%%%%%%%%%%%%%%%%%%%%%%%%%%%%%%%%%%
\section{Collaborations with others, and contributions to open source projects}

The following considerations apply when GRL employees or Sanger visiting/hosted 
scientists make substantial contributions to code maintained in collaboration with 
others outside the institute: 

\begin{itemize}
\item If the code is made publicly available under one of the standard acceptable open 
source licenses, or another license and header text that has been approved as 
described above, then again the line manager can approve contribution to the 
project, so long as a copyright statement is in place and the author's contribution 
is acknowledged. The copyright statement should include GRL as one of the 
copyright holders if a substantial fraction (say more than 25\% of the file) was 
written by a GRL employee or Sanger visiting/hosted scientist. 
\item If the code is developed as part of a research collaboration, and the software itself 
is a primary aim of the collaboration, then we would expect to require the 
software to be made widely available under an agreed license. If the 
collaboration is primarily about other matters, such as the generation and 
analysis of data, then it is acceptable for the software not to be distributed 
outside the collaboration, but even in this case a copyright statement and 
authorship attribution are required.
\item There may be cases where other arrangements are required, for example in 
collaborations with instrument manufacturers. In these cases a more formal 
agreement is required, which needs to be approved by the Director of Corporate 
Services who may consult the Wellcome Trust.
\item In some cases we use external software and either find and fix a bug, or make a 
useful modification, and want to feed that back to the author/copyright holder. 
Small bug fixes or enhancements of less than 100 lines or so of code are exempt 
from the provisions of this document. Larger contributions can also be made if 
authorised by an IC member. 
\end{itemize}


%%%%%%%%%%%%%%%%%%%%%%%%%%%%%%%%%%%%%%%%
% Signing-off 
%%%%%%%%%%%%%%%%%%%%%%%%%%%%%%%%%%%%%%%%
\section{Contributions to open source projects that require ``signing off''}

Some open source projects require code to be ``signed-off'' with a declaration that says ``I 
wrote this code and am authorized to contribute it to the project'' (see section 5 in 
\url{http://linux.yyz.us/patch-format.html} for an example). The Director of Corporate 
Services (\docsperson) can authorise code to be contributed in this fashion. 


%%%%%%%%%%%%%%%%%%%%%%%%%%%%%%%%%%%%%%%%
% Policy scope
%%%%%%%%%%%%%%%%%%%%%%%%%%%%%%%%%%%%%%%%
\section{Scope}

This policy applies to code you develop at work or at home when engaged in your 
work, whether on Sanger's or your own computer. If you wish to develop software in 
the general area of your work for which you wish to retain copyright with a view to 
commercialisation, you should discuss this beforehand with the Director of Corporate 
Services, or your IC member who would be expected to raise it with the Director of 
Corporate Services.


%%%%%%%%%%%%%%%%%%%%%%%%%%%%%%%%%%%%%%%%
% Commercialisation
%%%%%%%%%%%%%%%%%%%%%%%%%%%%%%%%%%%%%%%%
\section{Commercialisation of GRL software}

There are precedents where GRL software has been distributed under a commercial 
license, i.e. where payment has been received for use of the software. It is possible to 
distribute commercially software that is also available under an open source license, for 
example when someone wants to modify it and sell the modified version in a way 
incompatible with the open source license. If any GRL employee or Sanger 
visiting/hosted scientist wishes to pursue such a possibility, they should approach the 
Director of Corporate Services, keeping the IC informed. 


\appendix

%%%%%%%%%%%%%%%%%%%%%%%%%%%%%%%%%%%%%%%%
% Appendix A: source code boilerplate
%%%%%%%%%%%%%%%%%%%%%%%%%%%%%%%%%%%%%%%%
\section{Boilerplate headers for source code files}

The following are boilerplate headers that should be used, substituting DATES, PROGRAM-NAME, PROGRAM-AUTHOR and EMAIL as appropriate.

Note 1: The DATES in the copyright notice should be a comma-delimited list of years in which 
copyrightable contributions were made to the software. If and only if every year within a range 
of years really is a copyrightable year, you may choose to shorten such a range by using a `-' 
character between the first and last copyrightable years in the range. Multiple such ranges may 
be specified as elements of the comma-separated list, but it is also acceptable to list out each 
year individually. This line in the license should be updated every time a copyrightable 
contribution is made to the software. For example, if software was first written in 2005, then 
`2005' would be an appropriate value for DATES. If no substantial contributions were made 
during 2006, but new code was written in 2007, then DATES could be `2005, 2007' but should 
not be `2005-2007' since no copyrightable contributions were made in 2006. If a license file 
specified DATES as `2005, 2007-2009, 2011-2012' then the contributor would be equivalent to 
specifying DATES as `2005, 2007, 2008, 2009, 2011, 2012' and it indicates that no copyrightable 
contributions were made in 2006 or 2010. It is important not to assert copyright for years in 
which no copyrightable contributions were made since this could undermine our rights to 
enforce the copyright. 
 
Note 2: Including the email address (EMAIL) is optional. 

\subsection{GPL license (for software consisting of multiple source files)}

\begin{boilerplate}
\begin{verbatim}
Copyright (c) DATES Genome Research Ltd. 
 
Author: PROGRAM-AUTHOR <EMAIL> 

This file is part of PROGRAM-NAME. 

PROGRAM-NAME is free software: you can redistribute it and/or modify it under 
the terms of the GNU General Public License as published by the Free Software 
Foundation; either version 3 of the License, or (at your option) any later 
version. 
 
This program is distributed in the hope that it will be useful, but WITHOUT 
ANY WARRANTY; without even the implied warranty of MERCHANTABILITY or FITNESS 
FOR A PARTICULAR PURPOSE. See the GNU General Public License for more 
details. 
 
You should have received a copy of the GNU General Public License along with 
this program. If not, see <http://www.gnu.org/licenses/>. 
\end{verbatim}
\end{boilerplate}

\subsection{GPL license (for software consisting of only one source file)}
\begin{boilerplate}
\begin{verbatim}
Copyright (c) DATES Genome Research Ltd. 

Author: PROGRAM-AUTHOR <EMAIL> 

This program is free software: you can redistribute it and/or modify it under 
the terms of the GNU General Public License as published by the Free Software 
Foundation; either version 3 of the License, or (at your option) any later 
version. 

This program is distributed in the hope that it will be useful, but WITHOUT 
ANY WARRANTY; without even the implied warranty of MERCHANTABILITY or FITNESS 
FOR A PARTICULAR PURPOSE. See the GNU General Public License for more 
details. 

You should have received a copy of the GNU General Public License along with 
this program. If not, see <http://www.gnu.org/licenses/>. 
\end{verbatim}
\end{boilerplate}

\subsection{LGPL license (for software consisting of multiple source files)}
\begin{boilerplate}
\begin{verbatim}
Copyright (c) DATES Genome Research Ltd. 

Author: PROGRAM-AUTHOR <EMAIL> 

This file is part of PROGRAM-NAME. 

PROGRAM-NAME is free software: you can redistribute it and/or modify it under 
the terms of the GNU Lesser General Public License as published by the Free 
Software Foundation; either version 3 of the License, or (at your option) any 
later version. 
 
This program is distributed in the hope that it will be useful, but WITHOUT 
ANY WARRANTY; without even the implied warranty of MERCHANTABILITY or FITNESS 
FOR A PARTICULAR PURPOSE. See the GNU Lesser General Public License for more 
details. 
 
You should have received a copy of the GNU Lesser General Public License along 
with this program. If not, see <http://www.gnu.org/licenses/>. 
\end{verbatim}
\end{boilerplate}

\subsection{LGPL license (for software consisting of only one source file)}
\begin{boilerplate}
\begin{verbatim}
Copyright (c) DATES Genome Research Ltd. 

Author: PROGRAM-AUTHOR <EMAIL> 

This program is free software: you can redistribute it and/or modify it under 
the terms of the GNU Lesser General Public License as published by the Free 
Software Foundation; either version 3 of the License, or (at your option) any 
later version. 

This program is distributed in the hope that it will be useful, but WITHOUT 
ANY WARRANTY; without even the implied warranty of MERCHANTABILITY or FITNESS 
FOR A PARTICULAR PURPOSE. See the GNU Lesser General Public License for more 
details. 

You should have received a copy of the GNU Lesser General Public License 
along with this program. If not, see <http://www.gnu.org/licenses/>. 
\end{verbatim}
\end{boilerplate}

\subsection{AGPL license (for software consisting of multiple source files)}
\begin{boilerplate}
\begin{verbatim}
Copyright (c) DATES Genome Research Ltd. 

Author: PROGRAM-AUTHOR <EMAIL> 

This file is part of PROGRAM-NAME. 

PROGRAM-NAME is free software: you can redistribute it and/or modify it under 
the terms of the GNU Affero General Public License as published by the Free 
Software Foundation; either version 3 of the License, or (at your option) any 
later version. 

This program is distributed in the hope that it will be useful, but WITHOUT 
ANY WARRANTY; without even the implied warranty of MERCHANTABILITY or FITNESS 
FOR A PARTICULAR PURPOSE. See the GNU Affero General Public License for more 
details. 

You should have received a copy of the GNU Affero General Public License 
along with this program. If not, see <http://www.gnu.org/licenses/>.
\end{verbatim}
\end{boilerplate}

\subsection{AGPL license (for software consisting of only one source file)}
\begin{boilerplate}
\begin{verbatim}
Copyright (c) DATES Genome Research Ltd. 

Author: PROGRAM-AUTHOR <EMAIL> 

This program is free software: you can redistribute it and/or modify it under 
the terms of the GNU Affero General Public License as published by the Free 
Software Foundation; either version 3 of the License, or (at your option) any 
later version. 

This program is distributed in the hope that it will be useful, but WITHOUT 
ANY WARRANTY; without even the implied warranty of MERCHANTABILITY or FITNESS 
FOR A PARTICULAR PURPOSE. See the GNU Affero General Public License for more 
details. 

You should have received a copy of the GNU Affero General Public License 
along with this program. If not, see <http://www.gnu.org/licenses/>. 
\end{verbatim}
\end{boilerplate}

\subsection{Modified-BSD license}
\begin{boilerplate}
\begin{verbatim}
Copyright (c) DATES Genome Research Ltd. 

Author: PROGRAM-AUTHOR <EMAIL> 

Redistribution and use in source and binary forms, with or without 
modification, are permitted provided that the following conditions are met: 

   1. Redistributions of source code must retain the above copyright notice,
this list of conditions and the following disclaimer.
   2. Redistributions in binary form must reproduce the above copyright notice,
this list of conditions and the following disclaimer in the documentation
and/or other materials provided with the distribution.
   3. Neither the names Genome Research Ltd and Wellcome Trust Sanger Institute
nor the names of its contributors may be used to endorse or promote products
derived from this software without specific prior written permission.

THIS SOFTWARE IS PROVIDED BY GENOME RESEARCH LTD AND CONTRIBUTORS "AS IS" AND 
ANY EXPRESS OR IMPLIED WARRANTIES, INCLUDING, BUT NOT LIMITED TO, THE IMPLIED 
WARRANTIES OF MERCHANTABILITY AND FITNESS FOR A PARTICULAR PURPOSE ARE 
DISCLAIMED. IN NO EVENT SHALL GENOME RESEARCH LTD OR CONTRIBUTORS BE LIABLE 
FOR ANY DIRECT, INDIRECT, INCIDENTAL, SPECIAL, EXEMPLARY, OR CONSEQUENTIAL 
DAMAGES (INCLUDING, BUT NOT LIMITED TO, PROCUREMENT OF SUBSTITUTE GOODS OR 
SERVICES; LOSS OF USE, DATA, OR PROFITS; OR BUSINESS INTERRUPTION) HOWEVER 
CAUSED AND ON ANY THEORY OF LIABILITY, WHETHER IN CONTRACT, STRICT LIABILITY, 
OR TORT (INCLUDING NEGLIGENCE OR OTHERWISE) ARISING IN ANY WAY OUT OF THE USE 
OF THIS SOFTWARE, EVEN IF ADVISED OF THE POSSIBILITY OF SUCH DAMAGE.
\end{verbatim}
\end{boilerplate}

\subsection{GPL example}

\begin{boilerplate}
\begin{verbatim}
Copyright (c) 1995, 1996, 1997, 1998, 1999, 2000, 2001, 2002, 2003, 2004, 
2005, 2006, 2007, 2008, 2009, 2010, 2011, 2012 Genome Research Ltd. 
 
Author: A.N. Other <a.n.other@sanger.ac.uk> 
 
This file is part of uber_wtsi_program. 
 
uber_wtsi_program is free software: you can redistribute it and/or modify it 
under the terms of the GNU General Public License as published by the Free 
Software Foundation; either version 3 of the License, or (at your option) any 
later version.  

This program is distributed in the hope that it will be useful, but WITHOUT 
ANY WARRANTY; without even the implied warranty of MERCHANTABILITY or FITNESS 
FOR A PARTICULAR PURPOSE. See the GNU General Public License for more 
details. 
 
You should have received a copy of the GNU General Public License along with 
this program. If not, see <http://www.gnu.org/licenses/>. 
\end{verbatim}
\end{boilerplate}


%%%%%%%%%%%%%%%%%%%%%%%%%%%%%%%%%%%%%%%%
% Appendix B: documentation boilerplate
%%%%%%%%%%%%%%%%%%%%%%%%%%%%%%%%%%%%%%%%
\section{Boilerplate to include in documention}

The following boilerplate should be included somewhere in the 
documentation distributed with the software. It need not be at the start of a 
file (which would be a good place to reiterate the name of the software, to 
describe what it does, and to describe where to find additional documentation 
on installation and use of the software), but must be included somewhere in 
the documentation. 

\begin{verbatim}
The usage of a range of years within a copyright statement contained within 
this distribution should be interpreted as being equivalent to a list of years 
including the first and last year specified and all consecutive years between 
them. For example, a copyright statement that reads `Copyright (c) 2005, 2007-
2009, 2011-2012' should be interpreted as being identical to a statement that 
reads `Copyright (c) 2005, 2007, 2008, 2009, 2011, 2012' and a copyright 
statement that reads `Copyright (c) 2005-2012' should be interpreted as being 
identical to a statement that reads `Copyright (c) 2005, 2006, 2007, 2008, 
2009, 2010, 2011, 2012'.
\end{verbatim}


%%%%%%%%%%%%%%%%%%%%%%%%%%%%%%%%%%%%%%%%
% Document history
%%%%%%%%%%%%%%%%%%%%%%%%%%%%%%%%%%%%%%%%
\section*{Document history}

\begin{tabular}{|c|c|p{6cm}|}
\hline
\textbf{version (date)}	& \textbf{author}	& \textbf{changes} \\
\hline
v1 (11 Oct 2006)		& Richard Durbin	& \parbox{6cm}{
} \\
\hline
v2 (20 Sep 2007)		& Tim Hubbard		& \parbox{6cm}{
} \\
\hline
v3 (3 Dec 2009)		& Tim Hubbard		& \parbox{6cm}{
} \\
\hline
v4 (23 Dec 2010) 		& Tim Hubbard		& \parbox{6cm}{
clarifications: policy statement; section 1 (LGPL license description). \\
amendments: Appendix 3. (Modified-BSD license text) \\
} \\
\hline
v5 (4 May 2012)		& Tim Hubbard		& \parbox{6cm}{
clarifications: policy statement. \\
amendments: AGPL licenses added, Appendix 3. (boilerplate instructions) \\
} \\
\hline
v6 (18 July 2014)		& Joshua Randall	& \parbox{6cm}{
amendments: updated Director of Corporate Services. \\
formatting: converted to LaTeX, consolidated version/author history. \\
} \\
\hline
\end{tabular}


\end{document}
